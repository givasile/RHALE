\documentclass[final]{beamer}
%% Possible paper sizes: a0, a0b, a1, a2, a3, a4.
%% Possible orientations: portrait, landscape
%% Font sizes can be changed using the scale option.
\usepackage[size=a0,orientation=portrait,scale=1.3]{beamerposter}

\usetheme{gemini}
\usecolortheme{seagull}
\useinnertheme{rectangles}

% ====================
% Packages
% ====================

\usepackage[utf8]{inputenc}
\usepackage{graphicx}
\usepackage{bm, amsmath}
\usepackage{hyperref}
\graphicspath{{./../../DALE/ACML-poster}}
\hypersetup{
    colorlinks=true,
    linkcolor=blue,
    filecolor=magenta,      
    urlcolor=cyan,
    pdftitle={Overleaf Example},
    pdfpagemode=FullScreen,
    }
% \usepackage[most]{tcolorox}
\usepackage{wrapfig}
\usepackage{booktabs}
\usepackage{tikz}
\usepackage{pgfplots}

\newcommand{\xb}{\mathbf{x}}

% ====================
% Lengths
% ====================

% If you have N columns, choose \sepwidth and \colwidth such that
% (N+1)*\sepwidth + N*\colwidth = \paperwidth
\newlength{\sepwidth}
\newlength{\colwidth}
\setlength{\sepwidth}{0.03\paperwidth}
\setlength{\colwidth}{0.45\paperwidth}

\newcommand{\separatorcolumn}{\begin{column}{\sepwidth}\end{column}}

% ====================
% Logo (optional)
% ====================

% LaTeX logo taken from https://commons.wikimedia.org/wiki/File:LaTeX_logo.svg
% use this to include logos on the left and/or right side of the header:
\logoright{\includegraphics[height=5cm]{logos/athena-278x300.png}}
\logoleft{\includegraphics[height=4cm]{logos/dit-logo.png}}

% ====================
% Footer (optional)
% ====================

\footercontent{
	ECAI 2023, Krakow, Poland \hfill
	\insertdate \hfill
	\href{mailto:vgkolemis@athenarc.gr}{\texttt{vgkolemis@athenarc.gr}}
}
% (can be left out to remove footer)

% ====================
% My own customization
% - BibLaTeX
% - Boxes with tcolorbox
% - User-defined commands
% ====================
\input{custom-defs.tex}

%% Reference Sources
\addbibresource{refs.bib}
\renewcommand{\pgfuseimage}[1]{\includegraphics[scale=2.0]{#1}}

\title{RHALE: Robust and Heterogeneity-aware Accumulated Local Effects}

\author{Vasilis Gkolemis \inst{1, 2} \and Theodore Dalamagas \inst{1} \and Eirini Ntoutsi \inst{3} \and Christos Diou\inst{2}}

\institute[shortinst]{\inst{1} ATHENA Research Center \samelineand \inst{2} Harokopio University of Athens \samelineand \inst{3} Bundeswehr University of Munich}

\date{Octover 1, 2023}

\begin{document}
	
\begin{frame}[t]
	\begin{columns}[t] \separatorcolumn
		\begin{column}{\colwidth}
      \begin{alertblock}{TL;DR} \Large{\textbf{RHALE:} \textbf{Robust} and \textbf{Heterogeneity-aware} ALE}
        \begin{itemize}
          \item Robust: auto-bin splitting
          \item Heterogeneity: $\pm$ from the average
        \end{itemize}
        \vspace{10mm}
        \large{\textbf{keywords:} eXplainable AI, ALE, Heterogeneity, Feature Effect}
			\end{alertblock}

			\begin{block}{Motivation - ALE limitations}{}
        ALE (Apley et. al) is a SotA feature efect method but it has two limitations:
        \begin{itemize}
        \item it does not quantify the heterogeneity, i.e., deviation of the instance-level effects from the main (average) effect
          \item it is vulnerable to poor approximations, due to the bin-splitting step
        \end{itemize}
        \begin{figure}
          \centering
          \includegraphics[width=0.49\textwidth]{./../code/concept_figure/exp_1_ale_5_bins_0.pdf}
          \includegraphics[width=0.49\textwidth]{./../code/concept_figure/exp_1_ale_50_bins_0.pdf}
          \caption{Left: ALE approximation with a narrow bin-splitting (5 bins). Right: ALE approximation with a dense bin-splitting (50 bins)}
          \label{fig:acc-1}
        \end{figure}
        \begin{itemize}
        \item Both approximations are bad:
          \begin{itemize}
          \item Narrow bin-splitting hides fine-grain details
            \item Dense bin-splitting is noisy (low samples per bin rate)
          \end{itemize}
          \item No information regarding the heterogeneity
        \end{itemize}
			\end{block}


      %%%%%% DALE vs ALE
      \begin{defbox}{Simple approach: ALE + Heterogeneity}{}

        \begin{tcolorbox}[ams equation*, title=ALE main effect definition]
            f^{\mathtt{ALE}}(x_s) = \int_{x_{s,\min}}^{x_s} \underbrace{\mathbb{E}_{X_c|X_s=z}\left [f^s (z, X_c)\right ]}_{\mu(z)} \partial z
        \end{tcolorbox}

        \begin{tcolorbox}[ams equation*, title=ALE main effect approximation]
          \hat{f}^{\mathtt{ALE}}(x_s) =
          \Delta x \sum_k^{k_x} \underbrace{\frac{1}{|\mathcal{S}_k|}
            \sum_{i:\xb^i \in \mathcal{S}_k} [\frac{\partial
                f}{\partial x_s}(x_s^i, \bm{x^i_c})]}_{\text{bin effect}: \hat{\mu}(z)}
        \end{tcolorbox}

        \begin{tcolorbox}[ams equation*, title=ALE heterogeneity definition]
          \sigma(x_s) = \sqrt{\int_{x_{s,\min}}^{x_s} \underbrace{\mathbb{E}_{X_c|X_s=z}\left [ \left (f^s (z, X_c) - \mu(z) \right )^2 \right ] \partial z}_{\sigma^2(z)}}
        \end{tcolorbox}


        \begin{tcolorbox}[ams equation*, title=ALE heterogeneity approximation]
  \mathtt{STD}(x_s) = \sqrt{\sum_{k=1}^{k_x} (z_k - z_{k-1})^2 \underbrace{\frac{1}{|\mathcal{S}_k| - 1}
\sum_{i:\mathbf{x}^i \in \mathcal{S}_k} \left ( f^s(\mathbf{x}^i) -
  \hat{\mu}(z_1, z_2) \right )^2}_{\hat{\sigma^2(z)}}}
        \end{tcolorbox}

  \end{defbox}

      %%%%%% DALE vs ALE
      \begin{defbox}{Simple but wrong: ALE + Heterogeneity}{}

        \begin{figure}
          \centering
          \includegraphics[width=0.49\textwidth]{./../code/concept_figure/exp_1_rhale_5_bins_0.pdf}
          \includegraphics[width=0.49\textwidth]{./../code/concept_figure/exp_1_rhale_50_bins_0.pdf}
          \caption{Left: approximation with narrow bin-splitting (5 bins) and (Right) with dense-bin splitting}
          \label{fig:acc-1}
        \end{figure}

        \vspace{2mm}

        \begin{enumerate}
        \item Fixed-size bin splitting can ruin the estimation of the heterogeneity
        \end{enumerate}

  \end{defbox}

\end{column}

\separatorcolumn
\begin{column}{\colwidth}

  % DALE ACCURACY
  \begin{block}{RHALE: Robust and Heterogeneity-aware ALE}

    \begin{figure}
      \centering
      \includegraphics[width=0.95\textwidth]{./../code/concept_figure/exp_1_rhale_0.pdf}
      \caption{Narrow bins (\(K=40\)) \(\Rightarrow\) limited
        \(\frac{\text{samples}}{\text{bin}}\) \(\Rightarrow\) both plots are noisy }
      \label{fig:acc-1}
    \end{figure}

    \vspace{2mm}
    Simple but correct:
    \begin{itemize}
    \item Automatically finds the \textbf{optimal} bin-splitting
    \item Optimal $\Rightarrow$ best approximation of the average (ALE) effect
    \item Optimal $\Rightarrow$ best approximation of the heterogeneity 
    \end{itemize}
  \end{block}

      \begin{defbox}{Optimal bin-splitting}{}

        In the paper, we \textbf{formally prove}:  
        \begin{enumerate}
        \item the conditions under which a bin does not hide the finer details of the main effect
        \item the conditions under which a bin is an unbiased approximator of the heterogeneity

        \item that given (1) and (2), increasing bin size leads to a reduction in estimation variance
        \end{enumerate}

        \vspace{10mm}

        Based on the above, we formulate bin-splitting as an optimization problem and propose an efficient solution using dynamic programming.

  \end{defbox}

  
      \begin{alertblock}{Conclusion} In case you work with a differentiable model, as in Deep Learning, use RHALE to:
        \begin{itemize}
        \item quantify the heterogeneity of the ALE plot, i.e., the deviation of the instance-level effects from the average effect
        \item get a robust approximation of (a) the main ALE effect and (b) the heterogeneity, using automatic bin-splitting
      \end{itemize}
    \end{alertblock}

    \begin{block}{References}
      \begin{itemize}
      \item \large Paper repo: \href{git@github.com:givasile/RHALE.git}{git@github.com:givasile/RHALE.git}
      \item \large Personal site: \href{givasile.github.io}{givasile.github.io}
      \item \large Twitter: \href{https://twitter.com/givasile1}{twitter.com/givasile1}
      \end{itemize}
    \end{block}

    \begin{figure}
      \centering
      \includegraphics[width=0.49\textwidth]{./Publications.png}
    \end{figure}

\end{column} \separatorcolumn
\end{columns}

% \begin{columns}[t]\separatorcolumn
%     \begin{column}{1.3\colwidth}
%       \large
% 	\end{column} \separatorcolumn
% 	\begin{column}{0.7\colwidth}
% \end{column}
% \separatorcolumn
% \end{columns}


\end{frame}

\end{document}